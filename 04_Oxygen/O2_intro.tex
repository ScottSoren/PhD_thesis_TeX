
\begin{equation}
\ch{2 H2O -> O2 + 4 (H+ + e-)}\label{rxn:OER4}
\end{equation}

The oxygen evolution reaction (OER, rxn \ref{rxn:OER4}) is the source of most of the efficiency lost in water electrolysis\cite{Seh2017, Kibsgaard2019}, whether done in an alkaline electrolyzer cell (AEC)\cite{LeRoy1979, Dionigi2016b} or polymer electrolyte membrane electrolyzer cell (PEMEC)\cite{Carmo2013, Reier2017}. Since hydrogen produced by water electrolysis plays a central role in the fossil-fuel-free energy system and chemical industry outlined in Section \ref{sec:our_part}, there is a lot to win by improving oxygen evolution catalysis.

Oxygen evolution catalysts can be split into two groups, based on the pH, and thus which electrolyzer technology, they are able to operate at. AECs operate in concentrated hydroxide solution (high pH), whereas PEM electrolyzers use a solid polymer electrolyte membrane (PEM) with strongly acidic groups, and so the water splitting reactions effectively occur at low pH\cite{Carmo2013, Xiang2016}. These two technologies were described briefly in Section \ref{sec:our_part}. Here, I give a brief outline of oxygen evolution catalysts and related challenges associated with each of these technologies to motivate the EC-MS studies presented in this Chapter.

At the high potentials ($U>$ 1.23 V vs RHE) needed to drive the oxidation of water, metal oxides and hydrated metal oxides are virtually the only thermodynamically stable solids\cite{Pourbaix1966}. However, while many metals have a thermodynamically stable solid phase at high potential and high pH (alkaline electrolyte), almost all metals form a soluble species at high potential and low pH (acidic electrolyte). The fact that so many materials are unstable under OER conditions can make the accurate measurement of OER activity a challenge. Just measuring the electrochemical current might lead to an overestimation of the oxygen evolution activity, as Reaction \ref{rxn:OER4} might not account for all of the electrons transfered. A few examples of this from my PhD work are shown in Section \ref{sec:see_the_O2}.

The fact that most materials are not stable at high potential and low pH limits OER in acid to noble metal oxides, of which \ch{IrO2} and \ch{RuO2} are by far the most active\cite{Miles1976}. Even so, at least 200 mV of overpotential is required for reasonable current densities. Perhaps more importantly, both Ir and Ru are among the rarest elements on earth and among the elements produced in the least quantities - only approximately 4\cite{Babic2017} to 9\cite{Vesborg2012c} tons of Ir and 25 tons of Ru\cite{Vesborg2012c} are produced annually. All of this production is a biproduct of platinum production\cite{Vesborg2012c} and thus extremely inelastic to changes in demand. Of the two oxides, \ch{RuO2} is more active but considerably less stable, and so \ch{IrO2} is used in commercial PEM electrolyzers\cite{Carmo2013}. With the iridium loadings in current state-of-the-art PEMEC's, the entire global supply of iridium could add about 2 GW of hydrogen-producing capacity per year\cite{Babic2017}, which is clearly not on a scale of relevance to adding energy storage to the world's 20 TW of energy consumption\cite{Vesborg2012c}. The scarcity of these materials thus makes it essential to increase their mass-normalized activity and thus reduce the required loading. In Section \ref{sec:low_O2}, we measure the \ch{O2} from OER on \ch{RuO2} at record low overpotentials, in the hope that accurate measurement of activity at low overpotentials can provide fundamental insight to guide the design of more active catalysts for PEMEC's.

\begin{figure}[h]
	\includegraphics[width=1\textwidth]{04_Oxygen/fig/alkaline_TOF.png}
	\caption{Turn-over-frequency for state-of-the-art OER catalysts in alkaline electrolyte at 300 mV overpotential. From Paper \ref{Roy2018}.}
	\label{fig:alkaline_TOF}
\end{figure}

In contrast, the oxygen evolution catalysts used at high pH in AEC's need not be rare and expensive metals. Indeed, the industrially used catalyst is nickel (importantly, with impurities including iron)\cite{LeRoy1979, Xiang2016, Trotochaud2014a}. Nickel-iron oxy-hydroxide is also the most active catalyst on a turn-over-frequency (TOF) basis, as seen in Figure \ref{fig:alkaline_TOF}, taken from Paper \ref{Roy2018}. Turn-over-frequencies are notoriously difficult to calculate for oxygen evolution catalysts The calculation of this turnover frequency relies on the assumption that only the surface of the catalyst is active, which we base on isotope-labeling studies showing that oxygen within the catalyst is not evolved as \ch{O2}. These experiments are described in Section \ref{sec:lattice_O}. 

Such isotope-labeling studies are commonly used to establish the presence or lack of lattice oxygen involvement in the OER. This has often been described as a positive catalytic characteristic, facilitating an OER mechanism with higher rates\cite{Grimaud2017, Geiger2018}. However, such conclusions need to be made carefully, as the lattice-involving mechanism can be negligible when quantitatively compared with the conventional mechanism, and can sometimes be associated with degradation of the catalyst. A 























thorough and quantitative set of isotope-labeling experiments coupled with dissolution measurements is shown in Section \ref{sec:dissolving} for \ch{RuO2} and \ch{IrO2}. We plan to publish this work, together with that in section \ref{sec:low_O2}, in Paper \ref{Scott2019_RuO2}.

%It took a lot of trial and error to arrive at satisfactory techniques for sensitive and quantitative experiments definitively measuring lattice oxygen evolution. Much of Section \ref{sec:lattice_O} documents various things tried along the way. The time-pressed reader less interested in preliminary results and technique development could skip Section \ref{sec:lattice_O} and jump to Section \ref{sec:dissolving}, which includes the results our best lattice oxygen evolution experiments to date.

The scarcity and instability of the only available OER catalysts for PEMEC's begs the question whether they are actually worth researching, when AEC's are already an industrial technology. However, PEMEC's have a few distinct advantages over AEC's that are important for utilization of variable renewable energy\cite{Carmo2013}: (1) They can run more efficienctly due to the high conductivity of the solid electrolyte. (2) They have less \ch{H2} crossover, enabling them to run safely at a wider range of current densities. (3) They have faster load response, enabling them to better utilize intermittent renewable electricity when it is cheapest, that is, when the sun is shining and the wind is blowing. For these reasons, most experts expect PEMEC's to be the dominant water electrolysis technology by 2030\cite{Schmidt2017}. In the final Chapter of this thesis, Chapter \ref{ch:impact}, I will therefore try to estimate the impact of an incremental improvement in PEMEC efficiency on global \ch{CO2} emissions in order to get an idea of the impact of this PhD project.