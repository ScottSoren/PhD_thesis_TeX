
\section{Conclusion}\label{sec:conclusion}


\subsection*{The Project}

Chapter \ref{ch:Intro} described how greenhouse gas emissions need to be cut fast to avoid disasterous and worsening effects of climate change. To be in line with the ambitions of the Paris agreement, societies should cut emissions to half or less by 2030. In order for progress being made in wind and solar energy to drive sufficient overall emissions reductions, technological solutions are needed to keep the lights on when the wind isn't blowing and the sun isn't shining, and to spread the decarbonization to other sectors including transport and industry. Water electrolysis to make hydrogen for energy storage, fuel, and industrial reactant has the potential to help with all of these needs.

The goal of this PhD project has been to use in-situ techniques to better understand the fundamentals of a number electrocatalytic reactions. I've had the chance to work on the propene oxidation reaction (Paper \ref{Winiwarter2019}), the \ch{CO2} and \ch{CO} reduction reactions (Papers \ref{Scott2019_GIXRD}, \ref{Nitopi2019}), and the hydrogen evolution reaction (Paper \ref{Scott_Engstfeld2019}). However, this Thesis has focused on the oxygen evolution reaction (OER, Papers \ref{Roy2018} and \ref{Scott_Rao2019}). The OER, as the main source of efficiency loss in water electrolysis cells, and as a required counter-reaction for any other fuel-producing electrochemical process, is arguably the most important electrochemical reaction to understand and improve.

\subsection*{EC-MS and isotopes}

The tool I've used for most of the work in this PhD project is chip electrochemistry - mass spectrometry, described in Paper \ref{Trimarco2018}. This technique uses a microscopic sampling volume in a silicon microchip to saturate the electrolyte with inert or reactant gas, and to deliver any evolved gases to the vacuum chamber containing the mass spectrometer. This setup is highly sensitive and features well-controlled mass transport, making it ideal for fundamental studies. (Section \ref{sec:ECMS})

%Two model experiments - RHE potential measurement by saturating the electrolyte at a platinum electrode with \ch{H2}, and CO stripping - were shown first in isotopically natural electrolyte and then in isotope-labeled electrolyte. When a platinum electrode is simultaneously in contact with \ch{D2O}-labeled electrolyte and \ch{H2} and open-circuit potential, the evolved \ch{D2} gives a measure of the rate at which the forward and backwards reactions are happening at equilibrium. This experiment will, however, have to be repeated under conditions which are not dominated by mass transport. When \ch{C^{16}O} is oxidized in \ch{H2^{18}O}-containing electrolyte, it forms \ch{C^{16}O^{18}O}, as expected when one oxygen comes from \ch{CO} and one from \ch{H2O}. However, a \ch{C^{18}O2} signal follows, caused by oxygen switching between \ch{CO2} and \ch{H2O} via carbonic acid. The rate of change of the \ch{C^{18}O^{16}O} to \ch{C^{18}O2} ratio can be used to calculate the forward rate constant of the aqueous \ch{CO2} - carbonic acid equilibrium. 

Isotope-labeling, which enables the tracking of atoms, is a powerful strategy in catalysis, including electro-catalysis. An experiment that illustrates the possibilities of using chip EC-MS together with isotope labeling is one which confirms the Langmuir-Hinshelwood mechanism for CO electro-oxidation. If a sample with an \ch{^{18}O}-labeled surface oxide layer is reduced in \ch{C^{16}O}-saturated \ch{H2^{16}O} electrolyte, there is a transient of \ch{C^{16}O^{18}O} which is formed. This is when \ch{$*$ ^{18}O} on the surface of the oxide is reduced to \ch{$*$ ^{18}OH} which reacts with adsorbed \ch{$*$ ^{16}O}. This experiment can also confirm that a labeled oxide layer is still present in a sample if it is not evolved as \ch{^{16}O^{18}O} during OER.  (Section \ref{subsec:extraction})

\subsection*{Oxygen evolution experiments}

Checking for incorporation of lattice oxygen in evolved \ch{O2} by isotope-labeling is an experiment often used to gain insight into the OER mechanism on electrocatalyst materials (Section \ref{sec:lattice_O}). Such experiments, with a lot of trial and error, have also been an important part of my PhD project. From the literature and from the experience of my own studies, I conclude that the best way to do these experiments is to prepare the OER catalyst with full \ch{^{18}O} labeling and test in un-labeled electrolyte using a constant-current period of oxygen generation. A control sample which has the natural ratio should be tested first, to ensure that the m/z=32, m/z=34, and m/z=36 signals are as expected by the natural \ch{O2} isotopic distribution. Any excess m/z=34 or m/z=36 signal in the lattice oxygen evolution experiment should be quantified to get the absolute number of lattice oxygen atoms evolved. The electrolyte used for the isotope exchange experiment should be collected and analyzed by ICP-MS, so that the number of metal atoms dissolved can be compared quantitatively to the number of lattice oxygen atoms evolved. Only if lattice oxygen evolution exceeds metal dissolution can it be concluded that lattice oxygen exchange is part of the catalytic OER mechanism and not a degradation side-process. Finally, the isotopic composition of the electrocatalyst surface after oxygen evolution should be checked - in a truly catalytic process, \ch{^{16}O} should be incorporated into the catalyst. (Section \ref{sec:dissolving})

Along the way to this procedure, we concluded in Paper \ref{Roy2018} that nickel-iron based electrocatalysts in alkaline electrolyte do not evolve lattice oxygen. Using the recommended procedure, we showed that labeled ruthenium dioxide (\ch{Ru^{18}O2}) evolves some lattice oxygen but that it is most likely due to a degradation, and not a catalytic, mechanism. Labeled iridium dioxide (\ch{Ir^{18}O2}), on the other hand, does show a minor catalytic OER mechanism involving lattice oxygen exchange, but it amounts to only one ten-thousandth of the total evolved oxygen. This can only be detected because of the control and sensitivity enabled by the chip EC-MS method.

This sensitivity also allows us to observe oxygen evolution at low overpotential, where other process may mask it in the electrode current. On a high-surface-area ruthenium foam, we observe catalytic \ch{O2} evolution down to 1.29 V vs RHE, a record to the best of our knowledge, though the rates are tiny, i.e., a TOF of 10$^{-7}$ s$^{-1}$. The TOF of ruthenium foam together with a series of sputtered \ch{RuO2} films tested the same way fall on a shared TOF-vs-potential curve which we measured over seven orders of magnitude. This curve has a high and increasing potential dependence at very low overpotentials, which we hope will be useful input for rational OER catalyst design. (Section \ref{sec:low_O2})

\subsection*{Net \ch{CO2} impact}

But does all this work do any good? This is the question that the previous two Sections of this final Chapter have sought to answer. The question was simplified to consider only one metric: net \ch{CO2}; and to only consider one effect: an improvement in water oxidation efficiency. This PhD project has been a \ch{CO2}-intensive one due to a lot of traveling and high electricity consumption of the chip EC-MS setup. However, using results from an energy systems model of Europe for the year 2030, I found that the approximately 122 tons of \ch{CO2} emitted during this PhD are avoided in one year (the year 2030) if they can be attributed just an 0.03 mV improvement in the OER overpotential of electrolysis cells.

Since the results presented here have a non-zero chance of contributing to a breakthrough worth much more than that, I think this is not too optimistic an expectation value. Furthermore, this is without taking into account the value of the other projects and of the education, both my own education and the teaching and supervising I have done as part of this PhD program. I think it is therefore highly likely that this PhD project has done net good. There are, of course, lots of uncertainties involved - even in the aspect of my research that I chose for this calculation because I thought it the least uncertain. However, I think it could be good practice to consider, informally, how to maximize the net \ch{CO2} benefit of research projects.


