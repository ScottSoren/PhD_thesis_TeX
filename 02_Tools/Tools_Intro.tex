As described at the end of the previous Chapter, electrochemistry will play a central role in a steady-state civilization where all of the inputs to our energy infrastructure and chemical industries are renewable or closed-cycle. This will require the development of a wide range of new electrochemical processes and technologies, and the transition will be accelerated by increasing the efficiency and lowering the cost of existing electrochemical technologies, first and foremost water electrolysis. Central to these technologies are the electrode materials, or electrocatalysts, on which the electrochemical half-reactions in Table \ref{tab:EC_overview} take place. Research efforts around the globe have therefore flourished in recent years to develop new electrocatalyst materials and to improve the understanding of existing electrocatalyst materials. While, it can not be repeated enough, no realistic pace of progress in these efforts could remove the necessity of high and rising taxes on \ch{CO2} emissions, every bit of progress helps.

It is essential in electrocatalysis development to be sure that the reaction taking place is actually the desired reaction. This sounds obvious, but in electrochemistry it can be tempting to just measure the electrode current (the rate at which \ch{e-} are released or consumed) and not analyze the chemical products. Examples of when the electrode current can be misleading in oxygen evolution catalysis are given in Section \ref{sec:see_the_O2}. The need for product detection is even more important in electrochemical reactions which intrinsically have many possible products, such as the \ch{CO2} reduction reaction\cite{Nitopi2019}. In general, we need product detection to determine the \textit{Faradaic efficiency}, or the portion of the electrons transferred, for a specific reaction or product.

There are a number of product quantification methods including, for example, high pressure liquid chromatography (HPLC), gas chromatography coupled to temperature conductivity detection (GC-TCD) or flame ionization detection (GC-FID), nuclear magnetic resonance (NMR), or colorimetric methods which are all suitable for detecting various products of electrochemical reactions. These all have in common, though, that they typically require an electrochemical reaction to be run for some time to build up a concentration of a product. They are, in other words, ex-situ or batch product detection methods. Detecting electrochemical products after a batch reaction, while useful, often leaves out the information of how Faradaic efficiencies can change over time. For these reasons, we wish for an \textit{in situ}, i.e. continuous or equivalently ``real-time'', product detection. Mass spectrometry (MS), a readily available technology which is described in the start of Section \ref{sec:ECMS}, is a very useful tool in this regard because of its speed, ability to distinguish between molecules (\textit{chemical resolution}), and sensitivity.

Mass spectrometry, however, has the problem that it requires high vacuum, i.e. pressures less than 10$^{-6}$ mbar, to operate well. This is in stark contrast to electrochemical reactions, which are typically studied in aqueous media at ambient conditions. This leads to the \textit{interface problem}, to which we have developed a new solution, described in Paper \ref{Trimarco2018} and Subsections \ref{subsec:ECMS} and \ref{subsec:setups}. Since our solution involves interfacing the electrochemical environment and the vacuum chamber housing the mass spectrometer with a specially fabricated silicon microchip, we call it \textit{chip EC-MS}. Since it is the only electrochemistry - mass spectrometry technique used for this Thesis, it is sometimes simply referred to as EC-MS.

A pervasive idea in catalysis research is that an improved fundamental understanding of how and why the atoms move around on the surface of catalysts during the electrochemical reaction will enable the \textit{rational design} of more efficient, more stable, and less expensive catalysts. Electrocatalysis is no exception. In this effort, virtually no computational or experimental tool known to materials science has gone unturned in the quest to understand specific electrocatalysts and the fundamentals of electrocatalysis. 

Chip EC-MS has certain advantages, including sub-monolayer sensitivity, fast time response, and the ability to quickly dose reactive gases, that make it ideal for fundamental electrocatalytic studies. (It also happens to have some disadvantages with regards to more applied product detection, briefly described in \ref{subsec:disadvantages}, when compared to other techniques...) These advantages make it ideal for \textit{stripping experiments}, which probe the adsorbates on a surface by reactive desorption. This is a powerful type of experiment, in a word because the involvement of surface-adsorbed species is effectively the definition of catalysis. Papers \ref{Trimarco2018}, \ref{Winiwarter2019}, and \ref{Scott_Engstfeld2019} feature stripping experiments, and an example is included in \ref{subsec:examples}.

The extremely high sensitivity of chip EC-MS is based on the fact that every molecule of a gas (such as \ch{H2} or \ch{O2} by water splitting) produced at the electrode being studied goes through the chip and to the vacuum chamber. This also makes it a fantastic platform for absolute \textit{quantification} in mass spectrometry, in which a mass spectrometer signal is related not just to the concentration of an analyte, but to an absolute number of molecules of an analyte. This is the subject of Section \ref{sec:quantification}, which includes recommended procedures for using EC-MS as a generalized platform for quantitative mass spectrometry.

The final Section of this Chapter brings us to the juicy heart of this Thesis. 

Atoms are in general too small and too quick (when it's not extremely cold) to see them moving around. And there's the annoying problem that, in general, you can't tell two atoms of the same element apart, so it's impossible to keep track of them! If we wish, experimentally, to understand how atoms move around on an electrocatalytic surface, it is therefore very useful to be able to \textit{label} them. This can be done using \textit{isotopes}, which are versions of an element that have different number of neutrons, and thus different masses. Since chemistry is dominated by neutrons and electrons (the number of which defines the element), different isotopes of the same element behave (to a reasonable approximation) identically in electrocatalytic reactions. A mass spectrometer, though, can tell the difference! There are a number of exciting things to look at in electrocatalysis with isotope labeling and a sensitive enough setup. Two examples are given in Section \ref{subsec:isotopes}, and isotope labeling experiments are the primary focus of the following Chapter.

