

\chapter{The Right Tools to Answer the Right Questions}

\section{Electorchemistry - mass spectrometry}

\subsection{A brief history of MS and EC-MS}
Timeline for mass spectrometry

\begin{itemize}
	\item 1869 - fundamental - first observation of cathode rays (later identified as electrons) in electric discharge through vacuum tubes by Johann Wilhelm Hittorf.
	\item 1886 - fundamental - discovery of anode rays (ions) by Eugen Goldstein
	\item 1898 - fundamental - J. J. Thomson measures the m/z ratio of cathode rays (electrons)
	\item 1898 - fundamental - demonstration by Wilhelm Wein that anode rays are deflected much less and thus have much larger m/z than the better-known cathode rays.
	\item 1901 - fundamental - deflection by a magnetic field used to show that particles gain mass as their energy increases (relativistic mass).
	\item 1913 - fundamental - \ch{^{20}Ne} and \ch{^{22}Ne} seperated by J. J. Thomson and his student Francis Aston. Isotopes are discovered!
	\item 1918 - development - electron ionization described by Author Dempster
	\item 1919 - development - Francis Aston finishes the first full mass spectrometer. Ionization is done in a discharge tube, mass selection by crossed electric and magnetic fields, and detection by luminescence in a discharge tube.
	\item 1931 - development - cyclotron invented by Ernest O. Lawrence. Ions are accelerated in a spiral by a constant magnetic field and radio-frequency alternating electric fields.
	\item 1941 - application - a mass spectrometer based on the cyclotron, called the calutron, is used to separate isotopes of uranium for the Manhattan Project.
	\item 1943 - application - first successful commercialization of a mass spectrometer by the Consolidated Engineering Corporation
	\item 1946 - development - first time-of-flight mass analyzer developed by W. Stephens
	\item 1953 - development - first quadrupole mass spectrometer by Wolfgang Paul and Helmut Steinwedel
	\item 1956 - application - first GC-MS by Roland Gohlke and Fred McLafferty
	\item 1962 - application - first QMS sold to NASA for residual gas analysis
	\item 1963 - development - MIMS invented by Hoch and Kok for study with MS of volatiles in liquids
	\item 1966 - application - first use of mass spectrometry for peptide sequencing by Biemann, Cone, Webster, and Arsenault
	\item 1968 - development - electrospray ionization invented by Dole
	\item 1971 - development - first measurements of electrochemical products by Bruckenstein and Gadde
	\item 1980 - development - inductively coupled plasma (ICP) first used as an ion source for mass spectrometry by Robert Houk et al.
	\item 1983 - development - matrix-assisted laser desorption ionization (MALDI) developed by Tanaka, Karas, and Hillenkamp.
	\item 1984 - development - differential electrochemical mass spectrometry (DEMS) invented by Wolter and Heitbaum. 
	\item 2006 - development - online electrochemical mass spectrometry (OLEMS) developed by Marc Koper and coworkers
	\item 2009 - development - microreactor described by Henriksen et al for mass spectrometric analysis of the activity of a small amount of catalyst.
	\item 2015 - development - electrochemical microreactor described by Trimarco et al
	\item 2018 - development - membrane chip and EC-MS described by Trimarco and Scott
\end{itemize}


\subsection{Membrane chip EC-MS: working principle and implementation}\label{subsec:setups}

\textbf{\large The Sniffer Setup at DTU:}


\textbf{\large The ECMS-200A in Fuzhou:}


\textbf{\large Spectro Inlets:}



\subsection{Example experiments: RHE measurement and CO stripping}\label{sec:examples}


\section{Quantitative mass spectrometry: counting molecules}


\section{Isotope labeling: tracking atoms}


\subsection{A brief history of isotope labeling and its use in electrocatalysis}


\subsection{Example experiment: CO stripping in \ch{H2^{18}O} electrolyte}


