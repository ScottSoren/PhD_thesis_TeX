\chapter*{Foreword}

\section*{This Thesis}

I had a goal of writing a short Thesis. I must apologize to the reader that I have failed at that. I ran out of time, and so it is long.

The tone is informal, and I've tried to make it read like a story, with each Section setting the stage for the next. I haven't held back in writing my thoughts, describing the uncertainties and mistakes in the experiments presented here and even in the Papers, as well as things that I found especially exciting. I hope that this informal tone helps make it easy to read, so that it doesn't feel quite as long as the number of pages reveals it to be.

I have tried at the same time to organize the Thesis in such a way that it can be jumped around in: 

Some readers may not be especially interested in electrocatalysis or mass spectrometry or isotope labeling, but are instead drawn by the subtitle of this Thesis and by the first Section title, ``How much needs to change and how fast.'' Such readers only need Chapters 1 and 4, which are admittedly much briefer than they have any right to be given the richness of the subject matter. I saved Chapters 1 and 4 to write last, and that was a good idea, because otherwise I could have spent the whole time available learning about the climate crisis, emissions reduction policy, and energy systems modeling.

A reader who wants the full story but is pressed for time can skip Section 2.2 (which I think is valuable, but a side story) and all of the Subsections of 3.3 (which chronicle learning-by-doing) without missing too much. A reader who only wants a representative sample both of the fun and the finished isotope studies could just read Subsections 2.3.1 and 2.3.2, and then jump to Section 3.4. 

I've tried also to make this Thesis useful to potential readers. Some might only be interested in 2.2, which presents strategies for using electrochemistry-mass spectrometry as a platform for absolute quantification. There is an appendix describing some of the experimental procedures. I hoped to make a tutorial describing the use of \texttt{EC\_MS} data analysis python package which I developed and used for almost all of the results and plots presented in this Thesis, but did not get to it by the hand-in date, and so instead provide a link to where it will be.

If I had stuck to the original outline for this Thesis, it might have become even longer. I had planned consecutive chapters called ``Hydrogen'' and ``Oxygen'', but realized that I had run out of time to do the additional experiments necessary to justify writing the Hydrogen chapter. 

I had the opportunity to work with a lot of great people and become in a lot of projects during this PhD. This, and my poor control for length, made it essential to choose one topic to focus on. Isotope studies are the part of my PhD Project that I have felt the most personal ownership for, and thus which seemed most appropriate to express the medium of a PhD Thesis.

\section*{Acknowledgements}

I really appreciate all the fantastic people I've gotten to work together during this PhD project, including many who have graduated or gone on to other exciting things. During the first year of the PhD I was treated routinely to exciting and enlightening lunchtime conversations with Robert Jensen and Kenneth Nielsen. I also really enjoyed all the time I got to spend with Anders Bodin, Bela Sebok, Claudie Roy, and Daniel Trimarco, among others, and I'm glad our PhD periods overlapped. Daniel especially deserves credit for being a constant team-mate my first year, and for the visionary design and implementation of our electrochemistry-mass spectrometry system. I also want to shout out to Erlend Bertheussen, Thomas Hogg, and Thomas Maagaard, with whom I had a lot of fun going to beam times; and to Anna Winiwarter and Luca Silva, with whom I had a great time working on a propene oxidation side project.

The co-author lists on the attached Papers give a sense of how many great scientists I've gotten to work with. I appreciate the valuable conversations with Brian Seger, Peter Vesborg, Jakob Kibsgaard, Ifan Stephens, and Jan Rossmeisl, among others. One especially deserves thanks: my principle supervisor, Ib Chorkendorff. Ib has fostered a fantastic environment in our research group. He is open and honest, quick to give feedback, demanding but kind, always motivating, and full of scientific wisdom and experience. It has been an honor to work with and learn from him, and I'll be proud if I can ever lead a scientific group (or anything) nearly as well as he does.

In the wider world, I'm grateful for conversations with my neighbors Anjila Wegge and S{\o}ren Storgaard, who have taught me a lot from outside my physics/chemistry bubble about climate and energy. I also want to express my appreciation for Rachel Woods-Robinson, who has been an important source of inspiration and encouragement during this PhD project.

Finishing a PhD project is hard. Not just because of the thesis writing process, but also because of the life transition that it represents. I can't thank enough all of the people who have helped me get through the past several months with their support. Thanks to my friends and colleagues including but not limited to Anders Bjerrum, Cille Kvium, Thomas Hogg, Grith Martinsen, Anjila Wegge, Lars Hemmingsen, Stefan Laage, Alexander Krabbe, Valeria Magri, George Kozlov, Melih Erdem, Maike Sch{\"a}ffer, Chioma Nri, the rest of Konvencio and Admi, my band Los Kaminantes, Mille, Mathias, Anders, Marianne, and many more important and wonderful people. Thanks especially to my family who have cheered me on from across the pond. Mom, dad, brotherman Stoff, sistergirl Sofie: I don't think I could have made it through this without you.


\section*{Dedication}

Finally, I would like to dedicate this thesis to all of the climate activists who are pushing our reluctant societies into action. 

The work of scientists, engineers, entrepreneurs, and politicians in understanding the crisis and turning it around is courageous and essential. But people working in these professions, including myself, are also investing in a prestigious career which they know will likely open the door to other options if the funding or markets or votes moved to other priorities. Not all of us would necessarily take such a plan B, but we should still remember that activists in contrast in general have no such security when they sacrifice their sweat and more to push us in the right direction. The volunteers and organizers of Greenpeace, Extinction Rebellion, Ende Gel{\"a}nde, Fridays for Future, Den Gr{\o}nne Studenterbev{\ae}gelse, 350, and so many other organizations, as well as everyone who gets out on the streets for climate marches: You have my intense respect and gratitude. You are forcing the health of our planet to the top of our attention and calling our societies to action. 

Question 1.1 of this Thesis is \textit{How can we cut our emissions to half or less by 2030?}. It is largely because of activists that so many of us today are focused on this and similar questions. We need even more people on board finding and implementing the answers, so, activists, please keep up your work.


