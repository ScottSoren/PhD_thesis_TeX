
\begin{adjustwidth}{1cm}{1cm}

\chapter*{Abstract}

This thesis presents methods and results for isotope-labeling studies in oxygen evolution reaction (OER) electrocatalysis. The OER is an essential reaction for a transition to a fossil-fuel-free society. It is the main source of efficiency loss in the production of hydrogen by water electrolysis. Water electrolysis, in turn, is key for storing wind and solar energy and for using wind and solar electricity to decarbonize other sectors such as industry and transport. The first chapter puts this technological motivation in the context of the urgent need to mitigate climate change.

The second chapter describes and demonstrates the tools used in the isotope-labeling electrocatalysis studies. The primary tool is electrochemistry-mass spectrometry (EC-MS). The version of EC-MS used in this Thesis involves a silicon microchip to make the interface between the high vacuum of the mass spectrometer and the wet ambient environment of the electrochemistry experiment. The advantages of this technique, chip EC-MS, are high sensitivity, well-characterized mass transport, and the ability to dose reactant gases. 
%Two typical experiments in electrocatalysis - an reversible hydrogen electrode (RHE) potential measurement, and a CO striping experiment - are demonstrated on a platinum electrode through the window of chip EC-MS. 
%An Appendix describes the use of chip EC-MS as a platform for absolute quantification in mass spectrometry.
%Chip EC-MS can be used as a platform for absolute quantification in electrocatalysis, by which we mean determining from mass spectrometer signals the number of molecules entering a vacuum chamber. Three types of calibration to this end are described. The first is to use the current going to an electrochemical reaction to control the rate at which analyte molecules enter the mass spectrometer. The second is to calculate the rate at which a gas enters through the capillary of the chip, after its air flux has been determined via electrochemistry. The final method involves explaining the variation of measured sensitivity factors from the physics of mass spectrometry, and using this to predict the sensitivity to a new analyte - allowing generalized quantification of any analyte.

Isotope labeling studies are introduced with two examples. The first is an attempt to directly measure the hydrogen evolution exchange current density on platinum by electrochemical H-D exchange, which is however demonstrated to be mass-transport limited. The second is a set of CO stripping and CO oxidation experiments in labeled electrolyte (\ch{H2^{18}O}), which lead to a new way to probe the kinetics of the reaction of \ch{CO2} and \ch{H2O} to form carbonic acid. 
%A kinetic model is developed in order to use this observation to determine the forward rate constant for that equilibrium.

The third chapter is devoted to oxygen evolution electrocatalysis. The two main water electrolyzers, alkaline electrolyzer cells (AEC) and polymer electrolyte membrane electrolyzer cels (PEMEC), are briefly discussed in the context of the OER catalysts required. Then, the importance of measuring \ch{O2} is demonstrated with two examples in which the electrochemical current would overestimate the OER activity. This motivates the study with EC-MS of oxygen evolution on \ch{RuO2}, one of the only materials (together with \ch{IrO2}) that can catalyze the OER in the acidic environment of a PEMEC
%, but for which dissolution and capacitive charging can also contribute anodic current. 
.
Using isotope-labeled electrolyte to increase sensitivity, I measured the \ch{O2} produced by a series of \ch{RuO2} films and \ch{Ru} foams down to a record low 1.29 V vs RHE. All of these samples follow approximately the same trend of turn-over-frequency (TOF) vs potential with a very strong potential dependence at low overpotentials.

The involvement of lattice oxygen in the oxygen evolution mechanism has received a lot of research attention in recent years. This is investigated by preparing an OER catalyst with one isotope of oxygen (\ch{^{16}O} or \ch{^{18}O}) and measuring the isotopic composition of the \ch{O2} evolved in an electrolyte with a different isotopic composition than the catalyst. I present a comprehensive comparison of these studies, with views on the advantages and disadvantages of the methods employed. Using \ch{RuO2} and \ch{IrO2} samples as examples, and coupling the high sensitivity of chip EC-MS with dissolution measurements by inductively coupled plasma - mass spectrometry (ICP-MS) and surface isotopic characterization by ion scattering spectroscopy (ISS), I show that lattice oxygen evolution does not necessarily mean lattice oxygen exchange. In other words, an isotope signal in the oxygen evolved from a labeled OER catalyst does not necessarily imply that lattice oxygen plays an important catalytic role, which is demonstrated to be the case for sputter-deposited \ch{Ru^{18}O2}. 
%Indeed, small amounts of \ch{^{18}O} from \ch{Ru^{18}O2} are incorporated into the \ch{O2} evolved in un-labeled electrolyte, but a fully quantitative comparison reveals that the amount of \ch{^{18}O} evolved is always less than the amount of Ru dissolved, and ISS reveals that \ch{^{16}O} is not incorporated into the lattice, as it would be in a catalytic mechanism. 
%An appendix describes and critiques a number of earlier OER isotope-labeling studies from my PhD project, including those in Paper II.

In the last experiments presented in this thesis, CO oxidation is used as a probe for lattice oxygen reactivity. Under the right conditions, isotope-labeled oxygen from the catalyst is incorporated in the \ch{CO2} produced. These experiments can also be used as an in-situ proof that there is labeled oxygen at the surface of the electrocatalyst, for example after a negative result for lattice oxygen evolution. 

The final chapter ties the studies presented in this Thesis back to the motivation by estimating the amount of \ch{CO2} emissions avoided by a marginal improvement in electrolyzer efficiency. Using a simple model and literature-based assumptions about the future European energy system, I find that to achieve a one-year payback time on the \ch{CO2} costs of my PhD project by 2030 only requires that the results present here lead to an 0.03 mV improvement in the OER overpotential of electrolysis cells. 

The Chapters of this thesis represent a mix of published and as-of-yet unpublished work, and only a subset of the work done during my PhD project. The articles resulting from work during my PhD project are attached.

\clearpage

\chapter*{Resume}
Denne afhandling presentere



\end{adjustwidth}