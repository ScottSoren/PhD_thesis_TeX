\chapter{Papers from this PhD Project}
		
		\renewcommand{\thesection}{\Roman{section}}
		\titleformat{\section}{\normalfont\Large\bfseries}{Paper~\thesection}{1em}{}	
		
		\begin{flushleft}
		
		\section{Enabling real-time detection of electrochemical desorption phenomena with sub-monolayer sensitivity}\label{Trimarco2018}
		
		Daniel B. Trimarco*, Soren B. Scott*, Anil H. Thilsted, Jesper Y. Pan, Thomas Pedersen, Ole Hansen, Ib Chorkendorff, and Peter C.K. Vesborg. 
		
		*These authors contributed equally to this work
		
		\textit{Electrochimica Acta}, 2018, 268, 520-530
		
		DOI: 10.1016/j.electacta.2018.02.060
		
		\includepdf[pages=-]{Appendices/articles/Trimarco2018.pdf}
		
		
		
		
		\clearpage
		\section{Impact of nanoparticle size and lattice oxygen on water oxidation on \ch{NiFeO$_x$H$_y$}}\label{Roy2018}
		
		Claudie Roy*, Bela Sebok*, Soren B. Scott*, Elisabetta M. Fiordaliso, Jakob E. Sørensen, Anders Bodin, Daniel B. Trimarco, Christian D. Damsgaard, Peter C. K. Vesborg, Ole Hansen, Ifan E. L. Stephens, Jakob Kibsgaard and Ib Chorkendorff. 
		
		*These authors contributed equally to this work
		
		\textit{Nature Catalysis}, 2018, 1(11), 820-829 
		
		DOI: 10.1038/s41929-018-0162-x
		
		\includepdf[pages=-]{Appendices/articles/Roy2018.pdf}	
		
		
		
		
		
		\clearpage
		\section{Towards an Atomistic Understanding of Electrocatalytic Partial Hydrocarbon Oxidation: Propene on Palladium}\label{Winiwarter2019}
		
		Anna Winiwarter*, Luca Silva*, Soren B. Scott, Kasper Enemark-Rasmussen, Manuel Saric, Daniel B. Trimarco, Peter C. K. Vesborg, Poul G. Moses, Ifan E. L. Stephens, Brian Seger, Jan Rossmeisl, and Ib Chorkendorff.
		
		*These authors contributed equally to this work
		
		\textit{Energy and Environmental Science}, 12, 1055-1067, 2019.
		
		DOI:  10.1039/C8EE03426E
		
		\includepdf[pages=-]{Appendices/articles/Winiwarter2019.pdf}
		
		
		
		
		\clearpage
		\section{Absence of Oxidized Phases in Cu under CO Reduction Conditions}\label{Scott2019_GIXRD}
		
		Soren B. Scott, Thomas V. Hogg, Alan T. Landers, Thomas Maagaard, Erlend Bertheussen, John C. Lin, Ryan C. Davis, Jefferey W. Beeman, Drew Higgins, Walter S. Drisdell, Apurva Mehta, Brian J. Seger, Thomas F. Jaramillo, and Ib Chorkendorff.
		
		\textit{ACS Energy Letters}. 4, 803−804, 2019
		
		DOI: 10.1021/acsenergylett.9b00172 
		
		\includepdf[pages=-]{Appendices/articles/Scott2019.pdf}





		\clearpage
		\section{Progress and Perspectives of Electrochemical CO2 Reduction on Copper in Aqueous Electrolyte}\label{Nitopi2019}
		
		Stephanie A. Nitopi*, Erlend Bertheussen*, Soren Bertelsen Scott, Xinyan Liu, Albert K. Engstfeld, Sebastian Horch, Brian Seger, Ifan Stephens, Karen Chan, Christopher Hahn, Jens K. Nørskov, Thomas Jaramillo, and Ib Chorkendorff.
		
		*These authors contributed equally to this work
		
		\textit{Chemical Reviews}. 12, 7610-7672, 2019
		
		DOI: 10.1021/acs.chemrev.8b00705
		
		\vspace{2cm}
		
		\textbf{Note:}
		
		I became involved in this Paper, a review and perspective article, because I wrote a motivation for the \ch{CO2} reduction reaction as part of my master's thesis, which I expanded and adapted for this Paper. My primary contributions to the Paper were writing the introduction and producing a map of proposed mechanistic pathways of \ch{CO2} reduction. The Paper is 63 pages long, and so only the introduction is reproduced in this thesis.	
		
		
		\includepdf[pages=1-8]{Appendices/articles/Nitopi2019_ChemRev.pdf}
		\clearpage
		\vspace{5cm}
		
		%\includepdf[pages=27]{Appendices/articles/Nitopi2019_ChemRev.pdf}
		
		{
			\hspace{0pt}
			\vfill
			Stephanie A. Nitopi, Erlend Bertheussen, Soren Bertelsen Scott, Xinyan Liu, Albert K. Engstfeld, Sebastian Horch, Brian Seger, Ifan Stephens, Karen Chan, Christopher Hahn, Jens K. Nørskov, Thomas Jaramillo, and Ib Chorkendorff. Progress and Perspectives of Electrochemical CO2 Reduction on Copper in Aqueous Electrolyte. \textit{Chemical Reviews}. XXXX, XXX, XXX-XXX, 2019
			
			\vspace{1cm}
			
			\centering\Large
			Pages I-AX and BC - BK of this publication are not included in this thesis
			\vfill
			\hspace{0pt}
		}
		\clearpage
		
		\includepdf[pages=51-54]{Appendices/articles/Nitopi2019_ChemRev.pdf}
		


	
		\clearpage
		\section[In Preparation - Desorbing uphill: Anodic Hydrogen Evolution on Cu and Ru Electrodes]{Desorbing uphill: Anodic Hydrogen Evolution on Cu and Ru Electrodes}\label{Scott_Engstfeld2019}
		
		Søren B. Scott*, Albert K. Engstfeld*, Zenonas Yusys, Degenhart Hochfilzer, Nikolaj Knøsgaard, Daniel B. Trimarco, Peter C.K. Vesborg, R. Jürgen Behm, and Ib Chorkendorff
		
		\textit{In preparation}
		
		\includepdf[pages=-]{Appendices/articles/Scott_Engstfeld2019.pdf}
		
		
		
				
		\clearpage
		\section[In Preparation - Mechanistic study of oxygen evolution on \ch{RuO2} down to 60 mV overpotential]{Mechanistic study of oxygen evolution on \ch{RuO2} down to 60 mV overpotential}\label{Scott_Rao2019}
		
		\underline{Soren B. Scott}*, Reshma R. Rao*, Choongman M. Moon, Jakob E. S\o rensen, Jakob Kibsgaard, Yang Shao-Horn, and Ib Chorkendorff
		
		*These authors contributed equally to this work
		
		\textit{In Preparation}
		
		\vspace{5 mm}
		\begin{center}
		\textbf{Abstract}
		\end{center}

		Ruthenium and ruthenium dioxide are the most active catalysts for water oxidation in acidic electrolyte, but are not sufficiently stable at high current density for commercial applications. Herein, we apply an electrochemistry-mass spectrometry (EC-MS) system with unprecedented sensitivity to extend the activity trends of ruthenium-based electrodes to extremely low current densities, down to a turn-over frequency, (TOF) of 5x10$^{-4}\,$s$^{-1}$ at 60 mV overpotential. We show that the potential-dependence of the TOF, i.e the Tafel slope, goes through three regimes, which we explain by modeling the coverage of the reaction intermediates, confirming a mechanism in which *\ch{OOH} on the cus site of \ch{RuO2}(110) is stabilized with respect to *\ch{OH} by the oxygen at the bridge site. Both \ch{Ru} and \ch{RuO2} electrodes show improved stability in the small electrolyte volume of EC-MS setup compared to a conventional rotating disk electrode (RDE) setup, indicating that transport of dissolved species is important to the dissolution mechanism. Finally, by quantitative isotope labeling studies, we show that a small amount of lattice oxygen is evolved as \ch{O2}, but that the number of oxygen atoms evolved is small compared to the number of ruthenium atoms dissolved, indicating that lattice oxygen evolution is part of a dissolution mechanism, rather than the water oxidation mechanism.
	
		\vspace{1cm}
		\textbf{Note}
		
		This paper will include most of the results from Sections \ref{sec:low_O2} and \ref{sec:dissolving}.
		
		It will include Figures \ref{fig:RuO2_char}, \ref{fig:Reshma1_activities}, \ref{fig:Ru_foam_activity}, \ref{fig:Ru_TOF}, \ref{fig:EC-MS-MS_plots}, and \ref{fig:ISS_O}.
	

		\end{flushleft}